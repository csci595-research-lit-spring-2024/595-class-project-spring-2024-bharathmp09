%Two resources useful for abstract writing.
% Guidance of how to write an abstract/summary provided by Nature: https://cbs.umn.edu/sites/cbs.umn.edu/files/public/downloads/Annotated_Nature_abstract.pdf %https://writingcenter.gmu.edu/guides/writing-an-abstract
\chapter*{\center \Large  Abstract}
%%%%%%%%%%%%%%%%%%%%%%%%%%%%%%%%%%%%%%
% Replace all text with your text
%%%%%%%%%%%%%%%%%%%%%%%%%%%%%%%%%%%

In order to enchance MNIST classification this research focused to conduct an in-depth comparative analysis between two powerful architectures: the Denoising Autoencoder (DAE) and the Convolutional Neural Network (CNN). Both models are well-regarded for their distinct capabilities — DAE excels in unsupervised learning and feature extraction, while CNN is tailored for image-based tasks, particularly  classification. Motivated by the imperative to enhance the robustness of MNIST digit classification, this study navigates through the architectures of DAE and CNN, analyzing their impact on classification accuracy and noise resilience. The research extends beyond traditional performance metrics, delving into the interpretability of learned representations and assessing the models' ability to handle noisy input data. The research problem centers on deciphering how these architectures influence MNIST classification under varying levels of noise, providing insights into their respective strengths and limitations.  By addressing this research problem, the study aims to contribute to the advancement of image classification techniques, offering nuanced guidance on selecting models for MNIST classification tasks, especially in scenarios with noisy input data. The significance of this research lies in its potential to improve the resilience of classification models to real-world noise, thereby enhancing their applicability in practical settings. The findings are expected to benefit practitioners and researchers alike, guiding them in the selection and optimization of models for noise-affected MNIST classification scenarios.


~\\[1cm]
\noindent % Provide your key words
\textbf{Keywords:} Deep Learning, Denoising Autoencoder,
Convolutional Neural Networks, Classification, Accuracy.