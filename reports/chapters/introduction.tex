\chapter{Introduction}
\label{ch:into} % This how you label a chapter and the key (e.g., ch:into) will be used to refer this chapter ``Introduction'' later in the report. 
% the key ``ch:into'' can be used with command \ref{ch:intor} to refere this Chapter.
The realm of image classification, a complex perceptual task involving the categorization of objects from images, has witnessed significant advancements over the past decade. Referred to as the process of categorizing objects, image classification relies on the sophisticated analysis of multispectral data, utilizing the underlying multispectral pattern of each pixel as a quantitative basis for classification \cite{lillesand2015remote}. Notably, there has been a remarkable improvement in classification accuracy, reflecting the evolution of image classification models. In recent times, these models are increasingly applied across diverse fields, showcasing their versatility. Applications range from tasks such as handwritten digit recognition \cite{ahlawat2020improved} and  Vehicle detection and classification \cite{tsai2018vehicle}, deep learning approach to pneumonia classification \cite{stephen2019efficient}, and military object detection \cite{janakiramaiah2023military}. The existing models are broadly categorized into unsupervised and supervised modes, reflecting the diverse approaches employed in addressing the intricate challenges of image classification.



%%%%%%%%%%%%%%%%%%%%%%%%%%%%%%%%%%%%%%%%%%%%%%%%%%%%%%%%%%%%%%%%%%%%%%%%%%%%%%%%%%%
\section{Background}
\label{sec:into_back}
This research project is focused on enhancing the classification accuracy of the MNIST digit dataset, a widely used benchmark in the field of machine learning. The motivation behind this study is to address the importance of improving the robustness of MNIST digit classification, particularly in scenarios with noisy input data.

The project conducts an in-depth comparative analysis between two powerful neural network architectures: the Denoising Autoencoder (DAE) and the Convolutional Neural Network (CNN). Both models are chosen for their distinct capabilities, where the DAE excels in unsupervised learning and feature extraction, and the CNN is specifically tailored for image-based tasks, including classification.

The research delves into the impact of these architectures on classification accuracy and noise resilience, going beyond traditional performance metrics. It explores the interpretability of learned representations and evaluates the models' ability to handle noisy input data. Algorithms such as Denoising Autoencoders and Convolutional Neural Networks are the core components under investigation.

The project's significance lies in its potential to advance image classification techniques, providing nuanced guidance for selecting models in MNIST classification tasks, especially when dealing with noisy input data. The findings are expected to benefit both practitioners and researchers, offering insights that can guide the selection and optimization of models for real-world, noise-affected MNIST classification scenarios.
% \textbf{Cautions:} Do not say you choose this project because of your interest, or your supervisor proposed/suggested this project, or you were assigned this project as your final year project. This all may be true, but it is not meant to be written here.

%%%%%%%%%%%%%%%%%%%%%%%%%%%%%%%%%%%%%%%%%%%%%%%%%%%%%%%%%%%%%%%%%%%%%%%%%%%%%%%%%%%
\section{Research Question}
\label{sec:intro_prob_art}
The research question is thoroughly explored in this section.The central research problem revolves around understanding how the DAE and CNN architectures influence MNIST digit classification under varying levels of noise, providing valuable insights into their respective strengths and limitations. The main goal is to identify the aspects that affect accuracy, with a particular emphasis on the effects of noise levels.

%%%%%%%%%%%%%%%%%%%%%%%%%%%%%%%%%%%%%%%%%%%%%%%%%%%%%%%%%%%%%%%%%%%%%%%%%%%%%%%%%%%
\section{Aims and objectives}
\label{sec:intro_aims_obj}


\textbf{Aims:} This study's main goal is determine how well a CNN—which is optimised for image-related tasks—can classify handwritten digits in the MNIST dataset when compared to a DAE—which is specifically made for denoising and feature extraction. Also, provide insightful information on the advantages and disadvantages of both architectures on real world noisy images.

\textbf{Objectives:} This study's primary goals are to process data, apply varying level of noise on MNIST Dataset, train the denoising encoder and CNN's and hyper-tune parameters to get the best possible predicted accuracy. We shall assess both image classification models' performance at the conclusion.


%%%%%%%%%%%%%%%%%%%%%%%%%%%%%%%%%%%%%%%%%%%%%%%%%%%%%%%%%%%%%%%%%%%%%%%%%%%%%%%%%%%
\section{Solution approach}
\label{sec:intro_sol} % label of Org section
Briefly describe the solution approach and the methodology applied in solving the set aims and objectives.

Depending on the project, you may like to alter the ``heading'' of this section. Check with you supervisor. Also, check what subsection or any other section that can be added in or removed from this template.

\subsection{A subsection 1}
\label{sec:intro_some_sub1}
You may or may not need subsections here. Depending on your project's needs, add two or more subsection(s). A section takes at least two subsections. 

\subsection{A subsection 2}
\label{sec:intro_some_sub2}
Depending on your project's needs, add more section(s) and subsection(s).

\subsubsection{A subsection 1 of a subsection}
\label{sec:intro_some_subsub1}
The command \textbackslash subsubsection\{\} creates a paragraph heading in \LaTeX.

\subsubsection{A subsection 2 of a subsection}
\label{sec:intro_some_subsub2}
Write your text here...

%%%%%%%%%%%%%%%%%%%%%%%%%%%%%%%%%%%%%%%%%%%%%%%%%%%%%%%%%%%%%%%%%%%%%%%%%%%%%%%%%%%
\section{Summary of contributions and achievements} %  use this section 
\label{sec:intro_sum_results} % label of summary of results
Describe clearly what you have done/created/achieved and what the major results and their implications are. 


%%%%%%%%%%%%%%%%%%%%%%%%%%%%%%%%%%%%%%%%%%%%%%%%%%%%%%%%%%%%%%%%%%%%%%%%%%%%%%%%%%%
\section{Organization of the report} %  use this section
\label{sec:intro_org} % label of Org section
Describe the outline of the rest of the report here. Let the reader know what to expect ahead in the report. Describe how you have organized your report. 

\textbf{Example: how to refer a chapter, section, subsection}. This report is organised into seven chapters. Chapter~\ref{ch:lit_rev} details the literature review of this project. In Section~\ref{ch:method}...  % and so on.

\textbf{Note:}  Take care of the word like ``Chapter,'' ``Section,'' ``Figure'' etc. before the \LaTeX command \textbackslash ref\{\}. Otherwise, a  sentence will be confusing. For example, In \ref{ch:lit_rev} literature review is described. In this sentence, the word ``Chapter'' is missing. Therefore, a reader would not know whether 2 is for a Chapter or a Section or a Figure.

